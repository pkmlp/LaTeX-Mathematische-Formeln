
\documentclass[ 12pt, a4paper, parskip=full]{scrartcl}

\title{\LaTeX\ \\Mathematical formulas} 
\author{Peter Kessler}        
\date{\today}                 

\usepackage[utf8]{inputenc} 
\usepackage[english]{babel} 
\usepackage[T1]{fontenc}    

\usepackage{amsmath, amssymb}

\begin{document}

\maketitle
\thispagestyle{empty}

\pagebreak
\tableofcontents

\pagebreak
\section{Working with mathematical formulas}
LaTeX supports the insertion of mathematical formulas like hardly any other typesetting program. Formulas can be inserted either within a paragraph ("inline") or as a separate paragraph in a document.

\subsection{Inline Formulas}
Pythagoras says: If $a$ and $b$ are the 
cathetus and $c$ the hypotenuse, then $a^2+b^2=c^2$. 
Thus for the hypotenuse: $c=\sqrt{a^2+b^2}$.

\subsection{Separated Formulas}
Pythagoras says: If $a$ and $b$ are the cathets and $c$ is the hypotenuse, then holds: 
\begin{displaymath}
  a^2+b^2=c^2 
\end{displaymath}

Thus the following applies for the hypotenuse: 
\begin{displaymath}
  c=\sqrt{a^2+b^2} 
\end{displaymath}

\subsection{Separated Formulas Block}

Pythagoras says: If $a$ and $b$ are the cathets and $c$ is the hypotenuse, then holds: 
\begin{displaymath} % only one formula allowed
  a^2+b^2=c^2 
\end{displaymath} 
 
Thus the following applies for $a$, $b$ and $c$: 
\begin{displaymath} % only one formula allowed
  a=\sqrt{b^2+c^2} 
\end{displaymath} 
\begin{displaymath} % only one formula allowed
  b=\sqrt{c^2-a^2} 
\end{displaymath} 
\begin{displaymath} % only one formula allowed
  c=\sqrt{a^2+b^2} 
\end{displaymath} 

Thus the following applies for $a$, $b$ and $c$:
\begin{gather*} % allows a bunch of formulas
  a=\sqrt{b^2+c^2}  \\ 
  b=\sqrt{c^2-a^2}  \\
  c=\sqrt{a^2+b^2} 
\end{gather*}

Thus the following applies for $a$, $b$ and $c$:
\begin{align*} % allows a bunch of formulas
  a=\sqrt{b^2+c^2}  \\ 
  b=\sqrt{c^2-a^2}  \\
  c=\sqrt{a^2+b^2} 
\end{align*}

Thus the following applies for $a$, $b$ and $c$:
\[
  \begin{array}{c} % allows a bunch of formulas
    a=\sqrt{b^2+c^2}  \\ 
    b=\sqrt{c^2-a^2}  \\
    c=\sqrt{a^2+b^2} 
  \end{array}
\]

Many roads lead to Rome!

\subsection{Aligned and unnumbered Formulas}
There are two formulas that every child knows: 
% Mathematical formula set off and aligned,
\begin{align*}
  a^2 + b^2 &= c^2   \\  
  e &= m c^2
\end{align*}

\subsection{Aligned and numbered Formulas}
There are two formulas that every child knows: 
% Mathematical formula set off and aligned,
\begin{align}
  a^2 + b^2 &= c^2  \\  
  e &= m c^2  
\end{align}

\subsection{Aligned, numbered and referenced Formulas}
There are two formulas that every child knows: 
% Mathematical formula set off and aligned,
% with label for later referencing
\begin{align}
  a^2 + b^2 &= c^2     \label{eq.pythagoras}  \\  
  e &= m c^2           \label{eq.einstein}  
\end{align}

% Referencing formulas with label
Where \eqref{eq.pythagoras} is attributed to Pythagoras and \eqref{eq.einstein} to Albert Einstein.

\subsection{Un-/Aligned, numbered and referenced Formula Blocks}

Pythagoras says: If $a$ and $b$ are the cathets and $c$ is the hypotenuse, then holds: 
\begin{displaymath} % only one formula allowed
  a^2+b^2=c^2 
\end{displaymath} 

Thus the following applies for $a$, $b$ and $c$: 
\begin{gather} % allows a bunch of formulas
  a=\sqrt{b^2+c^2}  \label{pythagoras_1} \\ 
  \sqrt{c^2-a^2}=b  \nonumber \\
  c=\sqrt{a^2+b^2}  \nonumber
\end{gather}

Thus the following applies for $a$, $b$ and $c$: 
\begin{align} % allows a bunch of formulas
  a&=\sqrt{b^2+c^2}      \label{pythagoras_2} \\ 
  \sqrt{c^2-a^2}&=b      \nonumber \\
  c&=\sqrt{a^2+b^2}      \nonumber
\end{align}

This \eqref{pythagoras_1} and that \eqref{pythagoras_2} is only true if I have not made any mistakes in the transformation.

\end{document}
